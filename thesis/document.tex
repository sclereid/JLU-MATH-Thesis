\documentclass[openany,oneside]{book}

\usepackage{jluthesisUTF8}
%\usepackage{gbt7714}
\usepackage{amsmath}
\usepackage{fontspec}
\usepackage{graphicx}
\usepackage{tikz}
\usepackage{multirow}
\usepackage{lipsum}
\usepackage{graphicx}
\usepackage{tabularx}
\usepackage{listings}
\usepackage{pgf}
\usepackage{geometry}
%\usepackage{cite}

\usetikzlibrary{arrows}
\usepackage{xeCJK}


%opening
\hypersetup{
    pdftitle    = {your pdf title},
    pdfsubject  = {your pdf subject},
    pdfkeywords = {your pdf keywords},
    pdfauthor   = {your name}
}


\begin{document}

\frontmatter
\sloppy % 解决中英文混排的断行问题,会加入间距,但不会影响断行 ????

%
% 手动在长标题中利用 \par 输入断行,
\ctitle{一个因为某些机缘巧合变得冗长\par 而又枯燥的中文标题~~~~~~~~}
\etitle{O Serendipity, You Made This English Title~~~~~~\par Tediously Long~~~~~~~~}
                                       % 论文 内容提要
\cthesissummary{
    中文摘要。
}
%                                           % 关键词
\ckeywords{甲, 乙, 丙}

\ethesissummary {
    Engligh abstract goes here.
}

\ekeywords{a, b, c}

\cdate{2021年5月17日}
\cauthor{张~~三}

\makecover


\pagenumbering{Roman} 
%\pdfbookmark[0]{目~~~~录}{contents}

\tableofcontents
{\xiaosi}
%{\fontsize \fontsize{12.05pt}{14.45pt}\selectfont}
% 清除目录后面空页的页眉和页脚
\clearpage{\pagestyle{empty}\cleardoublepage}

%%% 正文
\mainmatter
\defaultfont                        % 正文使用默认字体,小四,宋体

\chapter{绪论}
\section{研究背景及意义}

研究背景及意义,可用cite去引用。\cite{marxapplication}

\lipsum

\section{研究现状及挑战}

研究现状及挑战。\cite{dean2008mapreduce, bzj2006constcurative}

\section{研究内容与论文结构}

研究内容与论文结构。

\chapter{一级标题}

一级标题

\section{二级标题}

{\hei 二级标题}

\subsection{三级标题}

三级标题

\chapter{例子}

\begin{quote}
    \begin{verse}
        我拿起烟斗装满叶烟,\\
        \hspace{1em}吞云吐雾消磨时间。\\
        人在那坐着,却飘走了思想------\\
        \hspace{1em}它飘向一幅画,晦黯而忧伤:\\
        \hspace{2em}那画告诉我这是多么相像,\\
        \hspace{2em}我和这烟斗一模一样。\\
        \vspace{1em}
        
        这喷香溢郁的烟斗同我相仿,\\
        \hspace{1em}都不过是尘芥不过是土壤;\\
        我最终也要归为尘灰。\\
        \hspace{1em}烟斗落地,没等听到声响,\\
        \hspace{2em}就已经拦腰摔断,不幸遭殃;\\
        \hspace{2em}相同的命运我也将承当。\par
        \vspace{1em}
        
        洁白的烟斗没有污斑,\\
        \hspace{1em}从未玷染,从未弄脏。\\
        \hspace{2em}可总有一天会命归无常, \\
        \hspace{1em}在草地下奄埋这副皮囊;\\
        \hspace{2em}我的肉体将变黑,一副晦暗相,\\
        \hspace{2em}就象这烟斗,若是它使用得经常。\\
        \vspace{1em}
        
        当烟斗被点燃,闪耀着火光,\\
        \hspace{1em}瞧,它们立刻就冒出青烟袅袅飘荡;\\
        \hspace{2em}轻烟散进空气,无处寻访,\\
        \hspace{1em}烟斗里便只剩有灰烬留藏。\\
        \hspace{2em}虚名也将消泯,正同那青烟一样,\\
        \hspace{2em}而躯体最终只不过化作上壤。\\
        \vspace{1em}
        
        吸烟时这种事也发生得经常:\\
        \hspace{1em}架上的塞烟器不翼而飞,给你添忙,\\
        \hspace{2em}这下你只好上手,虽然不很便当,\\
        \hspace{1em}可手指伸进烟锅,难免要被烫伤。\\
        \hspace{2em}既然烟斗里都有痛苦伏藏,\\
        \hspace{2em}地狱里的痛苦将会何其难当!\\
        \vspace{1em}
        
        由对烟斗的沉思,导向了别的玄想,\\
        \hspace{1em}沉溺在格致冥思中,\\
        \hspace{2em}也有益于身心的健康;\\
        \hspace{1em}这样吞云吐雾,心中着实舒畅,\\
        \hspace{2em}因此无论在国内、国外、在陆地或海洋,\\
        \hspace{2em}我都要一边抽我的烟斗,一边坚定对上帝的信仰。\\
    \end{verse}
\end{quote}



%最后设置格式,插入参考文献。
\defaultfont
\bibliographystyle{./gbt7714-author-year} %用GBT7714-2015格式进行排版。若未安装此拓展包导致报错,可以将\bibliographystyle改为plain,如下一行所示
%\bibliographystyle{plain}
\clearpage
\phantomsection
\addcontentsline{toc}{chapter}{参考文献}
\nocite{*} %展示所有的参考文献,即使正文中没有显式引用过。如果不需要,请注释掉这一行。
\bibliography{document}
%插入致谢
\chapter*{致 \qquad 谢}
\addcontentsline{toc}{chapter}{致谢}
\thispagestyle{empty}
本科生涯看似漫长却又一晃而过,回首走过的岁月,我感慨良多。从最初的论文选题、思路梳理到研讨交流、反复修改直至最终完稿,都离不开老师、同学和亲人们的支持和无私帮助,在此我要向他们表达我最诚挚的谢意。

...

求学生涯暂告段落,但求知之路却永无止境。我将倍加珍惜大学生活给予我的珍贵财富,不忘初心,砥砺前行!

\end{document}
